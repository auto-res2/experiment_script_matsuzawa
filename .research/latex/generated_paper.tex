\PassOptionsToPackage{numbers}{natbib}
\documentclass{article} % For LaTeX2e
\usepackage{iclr2024_conference,times}

\usepackage[utf8]{inputenc} % allow utf-8 input
\usepackage[T1]{fontenc}    % use 8-bit T1 fonts
\usepackage{hyperref}       % hyperlinks
\usepackage{url}            % simple URL typesetting
\usepackage{booktabs}       % professional-quality tables
\usepackage{amsfonts}       % blackboard math symbols
\usepackage{nicefrac}       % compact symbols for 1/2, etc.
\usepackage{microtype}      % microtypography
\usepackage{titletoc}

\usepackage{subcaption}
\usepackage{graphicx}
\usepackage{amsmath}
\usepackage{multirow}
\usepackage{color}
\usepackage{colortbl}
\usepackage{cleveref}
\usepackage{algorithm}
\usepackage{algorithmicx}
\usepackage{algpseudocode}
\usepackage{tikz}
\usepackage{pgfplots}
\usepackage{float}
\usepackage{array}  
\usepackage{tabularx}  
\pgfplotsset{compat=newest}


\DeclareMathOperator*{\argmin}{arg\,min}
\DeclareMathOperator*{\argmax}{arg\,max}

\graphicspath{{../}} % To reference your generated figures, see below.

\title{Test Paper for AMICT Evaluation}

\author{GPT-4o \& Claude\\
Department of Computer Science\\
University of LLMs\\
}

\newcommand{\fix}{\marginpar{FIX}}
\newcommand{\new}{\marginpar{NEW}}

\begin{document}

\maketitle

\begin{abstract}
This short paper presents a simplified evaluation of AMICT compared to a baseline. We include convergence and training loss analysis using selected visualizations.
\end{abstract}

\section{Introduction}
\label{sec:intro}
We briefly describe AMICT and its potential. This minimal version serves to verify LaTeX rendering and citation functionality.

\section{Related Work}
\label{sec:related}
text

\section{Background}
\label{sec:background}
text

\section{Method}
\label{sec:method}
We simulate training using synthetic datasets and monitor convergence and loss curves for both AMICT and the baseline.

\section{Experimental Setup}
\label{sec:experimental}
text

\section{Results}
\label{sec:results}
\begin{figure}[H]
  \centering
  \begin{subfigure}[b]{0.48\linewidth}
    \includegraphics[width=\linewidth]{images/convergence\_sequential\_parallel\_pair1.pdf}
    \caption{Convergence for sequential vs. parallel training.}
  \end{subfigure}
  \hfill
  \begin{subfigure}[b]{0.48\linewidth}
    \includegraphics[width=\linewidth]{images/convergence\_solver\_pair1.pdf}
    \caption{Solver behavior across initialization strategies.}
  \end{subfigure}

  \vspace{0.5em}

  \begin{subfigure}[b]{0.48\linewidth}
    \includegraphics[width=\linewidth]{images/training\_loss\_base\_pair1.pdf}
    \caption{Baseline model loss curve.}
  \end{subfigure}
  \hfill
  \begin{subfigure}[b]{0.48\linewidth}
    \includegraphics[width=\linewidth]{images/training\_loss\_taa\_pair1.pdf}
    \caption{AMICT training loss.}
  \end{subfigure}
  \caption{Evaluation of convergence patterns and training loss trends.}
\end{figure}

These figures illustrate the convergence pattern for sequential versus parallel training and the solver behavior under different initialization strategies, as well as the loss curves for both the baseline and AMICT models. The results demonstrate expected trends and help validate the evaluation pipeline.\newline
\citep{Xu2023}

\section{Conclusions and Future Work}
\label{sec:conclusion}
This test confirms that key figures render properly, and inline citations work as expected.

This work was generated by \textsc{AIRAS} \citep{airas2025}.

\bibliographystyle{iclr2024_conference}
\bibliography{references}

\end{document}
